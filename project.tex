\documentclass[12pt,a4paper,titlepage,twoside]{article}
    \usepackage[margin=1in,footskip=0.25in]{geometry}
    \usepackage[latin1]{inputenc}
    \usepackage{amsmath}
    \usepackage{multirow,tabularx}
    \usepackage{amsfonts}
    \usepackage{amssymb}
    \usepackage{makeidx}
    \usepackage{graphicx}
    \usepackage{pgfgantt}
    \usepackage{array}
    \usepackage{longtable}
    \usepackage{fullpage}
    \setcounter{tocdepth}{1}
    
    \title{Remoulding an existing home into a Smart Home}
    
    \begin{document}
        \begin{titlepage} 
            \newcommand{\HRule}{\rule{\linewidth}{0.5mm}}
            \centering 
            \textsc{\Large UEE693}\\[0.5cm] 
            \textsc{\large Capstone Project}\\[0.5cm]
            \HRule\\[0.4cm]
            {\huge\bfseries Remoulding an existing house into a Smart Home}\\[0.4cm]
            \HRule\\[1.5cm]
            \begin{minipage}{0.5\textwidth}
                \begin{flushleft}
                    \large
                    \textit{Investigating Group}\\[0.25cm]
                    101504114 Satyam Kumar\\
                    101504119 Shubham Gupta\\
                    101504120 Stuti Sidhu\\
                    101504122 Swanav\\
                \end{flushleft}
            \end{minipage}
            ~
            \begin{minipage}{0.4\textwidth}
                \begin{flushright}
                    \large
                    \textit{Supervisor}\\
                    Dr. Mukesh Singh
                \end{flushright}
            \end{minipage}
            \vfill\vfill\vfill
            {\large 1819ODDSEM}		
            \vfill\vfill
            \includegraphics[width=0.3\textwidth]{logo.png}\\
            Electrical and Instrumentation Engineering Department\\
            Thapar Institute of Engineering and Technology
        \end{titlepage}
    
        \pagenumbering{roman}
        \tableofcontents
        \newpage
        \pagenumbering{arabic}
    
        %Aim: 2-3 sentences
        \section{Aim}
            The project aims to increase the energy efficiency of a household by converting primitive devices into smart devices and managing them using a self-evolving deep learning model.
        
        %Literature Survey: Follow any IEEE Transaction papers and write the literature survey. From this literature Survey find the motivation to do this work.  (2 Pages)
        \section{Literature Survey}
            \subsection{National Status Review}
            \begin{itemize}
                \item Archana N. Shewale, \textbf{"Renewable Energy Based Home Automation System Using ZigBee"}, International Journal of Computer Technology and Electronics Engineering (IJCTEE), 2015\\ \\
                Archana N. Shewale describes the methodology of renewable energy based home automation in which two things are consider one is energy consumption and another is energy generation. In this, ZigBee is used for monitoring energy consumption of home equipment and power line communication (PLC) is used to monitoring energy generation.
                \item S. Anusha, \textbf{"Home Automation using ATmega328 Microcontroller and Android application"}, International Research Journal of Engineering and Technology(IRJET), 2015\\ \\
                S. Anusha describes the design and development of a remote household appliance control system using ATmega328 microcontroller and android mobile through GSM technology.
                \item J. Chandramohan, \textbf{"Intelligent Smart Home Automation and Security System Using Arduino and Wi-Fi"}, International Journal of Engineering and Computer Science, 2017\\ \\
                J. Chandramohan provides a low cost-effective and flexible home control and monitoring system with the aid of an integrated micro-web server with internet protocol (IP) connectivity for access and to control of equipment and devices remotely using Android-based smartphone application. generation.
            \end{itemize}
            \subsection{International Status Review}
            \begin{itemize}
                \item Debraj Basu, \textbf{"Wireless Sensor Network Based DSAda Smart Home: Sensor Selection, Deployment and Monitoring"}, IEEE, 2013\\ \\
                Debraj Basu details the installation and configuration of unobtrusive sensors in an elderly person?s house - a smart home in the making - in a small city in New Zealand. The overall system is envisaged to use machine learning to analyze the data generated by the sensor nodes.
                \item Byeongkwan Kang, \textbf{IoT-based monitoring system using tri-level context making model for smart home services"}, IEEE International Conference, 2015\\ \\
                Kang discusses about acquisition and analysis of sensor data which are going to be used across smart homes. It proposed an architecture for extracting contextual information by analysing the data acquired from various sensors and provide context aware services.
                \item Jeya Jeya Padmini, \textbf{"Effective Power Utilization and Conservation in Smart Home Using IoT"}, IEEE International Conference, 2015\\ \\
                Jeya Jeya Padmini discusses about effective power utilization and conservation in smart homes using IoT. It uses cameras for recognizing human activities through image processing techniques.
                \item Pranay P. Gaikwad, \textbf{"A Survey based on Smart Home System Using Internet of Things"}, IEEE International Conference, 2015\\ \\
                Pranay P.Gaikwad discusses about challenges and problems arise in smart home systems using IoT and propose possible solutions.
            \end{itemize}
        
        
        %Need Analysis: 4-5 sentences
        \section{Need Analysis}
        The currently available solution and research exhibit these features
        \begin{itemize}
            \item The existing market solutions provide on/off switching of the devices. The available solutions provide comparatively lower energy efficiency and also require human intervention to achieve desired conditioning of the environment.
            \item Some of them also use unsuitable technologies like Power Line Communications, Bluetooth and Ethernet etc. From the present analysis of the existing solutions for automating a room environment, the technologies used suffer from a number of drawbacks. So, these protocols are limited in functionalities when used by an end user in real life.
        \end{itemize}
        
        %Objectives: 3 objectives with bullet points
        \section{Objectives}
            \begin{itemize}
                \item To detect changes in the environment of rooms using different sensors.
                \item To develop plug and play device control techniques to manage home appliances.
                \item To optimize energy consumption to increase the efficiency of the household.
                \item To design a central processing hub for management of modules and execution of control algorithms.
            \end{itemize}
        
        %Novelty: 2 sentences
        \section{Novelty}
            Higher levels of controllability can be achieved through individual device level control such as for an AC the temperature, modes, swing, fan speed etc. can be controlled. Load optimization will be carried out from the evaluation of the usage patterns of the individual devices thus increasing the energy efficiency. 
        
        %Methodology: Detailed work packages and their associations (2 pages with proper flow diagram)
        %Each methodology should be linked with objective and deliverables
        %For example: Objective 1 is achieved by Work package 1. Work package 1 has three methodology and this work package has 1 deliverables.
        \section{Methodology}
        \subsection{Sensor aggregator module development}
        Sensors are collected in a single unit to form an interconnection among the sensors for efficient analysis and evaluation of aggregated data by the controlling unit.
        
        \subsubsection{Selection of sensors}
        As a first step, we need to figure out all the environment variables relevant to home automation system. These are the conceived energy based equipments which will be optimized :
        \begin{itemize}
            \item Lights
            \item Fans
            \item Air Conditioners
        \end{itemize}
        Now based upon the above identified appliances, the environment variables under observation will be
        \begin{itemize}
            \item Infrared Radiations - To detect the presence of human being in or around the target area.
            \item Temperature - To detect the temperature of the surrounding room environment.
            \item Illumination - The information provided by this sensor will be used by the system to, adjust the illumination accordingly to the desired level.
        \end{itemize}
        The choice of sensors based on the identified parameters will be done considering the room specifications and the accuracy and range provided by the sensor unit.
        
        \subsubsection{Aggregation of sensors}
        Based upon the appliances present inside a particular room, specific sensors will be aggregated to form a module of sensors and it will be installed inside every room. This will help in collecting optimum data and reduce complexity in the installation as well.\\
        
        The module formulated will be a combined unit containing sensors and circuitry for controlling an entire room. All such modules will form a network through the Home Automation Unit. This module will act as a local control placed centrally on the ceiling providing ease of access to every part of the room through a single unit.
        
        \subsection{Plug and Play actuator development}
        To introduce software based control, microcontroller based actuators modules are to be attached either in the power supply lines of the devices (fan and lighting system control) or must be directly attached to the device itself (Air Conditioner control).
        
        \subsubsection{Study of control methodology}
        The development of actuator modules designed for each device to be automated will be using plug and play methods so as to convert existing devices into smart ones. The actuators will operate and automate the appliances based on the commands received from the microprocessor.
        \subsubsection{Actuator circuit design}
        \begin{enumerate}
            \item Air Conditioner
            The plug and play module of the air conditioner will be equipped with IR blasters. It will then receive the signal and relay this to the receiving end to automate the air conditioner accordingly.
            \item LED Lights
            The use of pulse width modulated signals that drives a control switch (say MOSFET) to switch the LEDs accordingly thereby altering the intensity.
            \item Fans
            Speed regulation of fans will be done using the combination of  Diac and Triac to reduce the energy losses that were occurring by the use of conventional voltage controller.
        \end{enumerate}
        Now the designed circuit of actuator modules is designed in such a way that there would be no need to interfere with the existing circuitry of the appliances with a number of iteration done, we will achieve the most desired place of installation of the actuator modules.
        \subsection{Energy optimization algorithm formulation}
        \subsubsection{Study of decision influencing parameters}
        \subsubsection{Design of a deep learning model}
        \subsection{Home Automation Unit design}
        \subsubsection{Design of the software for the Home Automation System}
        All the sensor aggregator modules and the retrofit modules need a central hub for their communication and management. The HAS will receive the inputs from the sensors and give an optimized output to the modules after processing it using the aforementioned algorithms. The HAS will run atop it providing functionalities like adding new devices, removing devices, defining rooms, adding sensor aggregator units and plug and play smart device conversion modules.
        \subsubsection{Development of a user interface to enable user interaction with the system}
        Applications will be deployed on the market places for the most common mobile platforms, App Store for iOS and Play Store for Android. A cloud-based web app will also be deployed for users to operate from laptops and desktops. The user will be able to use these apps to connect directly to the server hosted on the Home Automation Unit in their homes without any middleware services ensuring their security and privacy.
        
        
        %Deliverables:  3 deliverables with bullet points.
        \section{Deliverables}
            \begin{itemize}
                \item Sensor modules to collect room environment parameters.
                \item Plug and Play actuator modules to control household devices.
                \item Load Optimization Algorithm based on deep reinforcement learning models
                \item Home Automation Unit to manage the devices and execute control algorithms and associated mobile applications.		
            \end{itemize}
        
        %Subject code and subject name: Link the portion of course subjects which you have studied in the course work ( The subject should include basic sciences, mathematics and engineering)
        \section{Associated Subjects}
            \begin{center}
                \begin{description}
                    \item [UTA007 Computer Programming-I] 
                    Fundamentals of functional programming
                    \item [UTA009 Computer Programming-II] 
                    Fundamentals of object oriented programming
                    \item [UTA011 Engineering Design-III] 
                    Embedded system design using sensors and micro-controllers
                    \item [UEE301 Direct Current Machines and Transformers]
                    Study of transformers
                    \item [UEE505 Analog and Digital Systems] 
                    Study of analog and digital systems and their interoperability.
                    \item [UEE401 Alternating Current Machines] 
                    Study of the single phase induction motors
                    \item [UEE504 Power Electronics] 
                    Use of switches to control device output in an efficient manner
                    \item [UEI404 Digital Signal Processing Fundamentals] 
                    Use of DSP techniques to process signal from sensors and other devices
                    \item [UEI609 Fundamentals of Microprocessors and Microcontrollers] 
                    Fundamentals of Assembly Programming and Embedded system design using micro-controllers 
                    \item [UEI501 Control Systems] 
                    Use of closed loop systems to eliminate errors in a system 
                    \item [UEE801 Electric Drives] 
                    Application of Power electronics to control AC Drives
                \end{description}
            \end{center}
        
        %Interdisciplinary work: Define the interdisciplinary work in the project.
        \section{Interdisciplinary Works}
        This project consists of extensive multidisciplinary efforts.
        The Load Optimization algorithm will be generated using a Deep Reinforcement Learning model which is primarily a topic of interest in Computer Science.
        The wireless communication among the sensors, devices and the HAU are subjects of Electronics and Communication Engineering.\\
        
        %Software used: (Available in the TU and open source: Try to use  the software which are available in the department)
        \section{Software Used}	
            Provided by college
            \begin{itemize}
                \item MATLAB
                \item LabVIEW
                \item Multisim and Ultiboard
            \end{itemize}
            Open Source
            \begin{itemize}
                \item Tensorflow
                \item Node.js
            \end{itemize}
        %Hardware to be used: (Details specifications and purpose of each equipment)
        \section{Hardware Used}
            \begin{itemize}
                \item PCB Prototyping Machine
                \item Digital Storage Oscilloscope
                \item Function Generator
                \item Multimeter
                \item Soldering station
                \item Mobile Phone
            \end{itemize}
        
        %A) Engineering standards 
        %B) Realistic constraints
        %List all the standards and constraints used in the design. 
        \section{Real-world compliance}
            \subsection{Engineering Standards}
            \subsection{Realistic Constraints}
        
        \newpage
        %Distribution of work among group members and tentative Gantt chart.
        \section{Work Schedule}
        {\centering
            \begin{ganttchart}[expand chart=\textwidth]{1}{12}
                \gantttitle{2018}{12} \\
                \gantttitlelist[]{1,...,12}{1} \\
                \ganttbar{Literature Survey}{4}{4} \\
                \ganttgroup{WP 1}{4}{6} \\
                \ganttbar{WP1.1}{4}{4} \\
                \ganttlinkedbar{WP1.2}{5}{6} \ganttnewline
                \ganttmilestone{Completion}{6} \ganttnewline
                \ganttgroup{WP 2}{4}{6} \\
                \ganttbar{WP2.1}{4}{5} \\
                \ganttlinkedbar{WP2.2}{5}{6} \ganttnewline
                \ganttmilestone{Completion}{6} \ganttnewline
                \ganttgroup{WP 3}{4}{7} \\
                \ganttbar{WP3.1}{4}{7} \ganttnewline
                \ganttmilestone{Completion}{7} \ganttnewline
                \ganttgroup{WP 4}{8}{11} \\
                \ganttbar{WP4.1}{8}{11} \\
                \ganttbar{WP4.2}{10}{11} \ganttnewline
                \ganttmilestone{Completion}{11} \ganttnewline
                \ganttbar{Integration}{12}{12}
                \ganttlink{elem2}{elem3}
                \ganttlink{elem6}{elem7}
                \ganttlink{elem9}{elem10}
                \ganttlink{elem12}{elem14}
                \ganttlink{elem13}{elem14}
                
                \ganttlink{elem3}{elem15}
                \ganttlink{elem7}{elem15}
                \ganttlink{elem10}{elem15}
                \ganttlink{elem14}{elem15}
            \end{ganttchart}
        }
        \newpage
        
        %Proposed budget (Items to be purchased):  Items, specifications, quantity, amount, company name, reasons in brief about its use (This should be in tabular form).
        \section{Proposed Budget}
        \begin{center}
            \setlength\extrarowheight{2pt}
            \begin{longtable}{| m{3cm} | m{3cm}| m{1.7cm} | m{1.5cm} | m{2.7cm} | m{3cm} |}
                \hline
                \textbf{Item}&\textbf{Specification}&\textbf{Quantity}&\textbf{Amount}&\textbf{Manufacturer}&\textbf{Justification}\\
                \hline
                Raspberry Pi 3 Model B&
                64-bit ARM-v8 processor\newline1.2GHz\newline1GB RAM&
                3&
                7,185&
                Raspberry Pi Foundation&
                Central Processing Hub\\
                \hline
                TP-Link Archer C20 AC750 Wireless Dual Band Router&
                &
                1&
                1,499&
                TP-Link&
                Wifi Router to establish the network around the house\\
                \hline
                esp8266 Wifi Board&
                GPIO\newline SPI\newline SDIO\newline I2C&
                5&
                1,045&
                Espressif Systems&
                Wifi Board for wireless modular connectivity\\\hline
                ATTiny Microcontroller&
                1 timer\newline2 PWM Channel\newline 4 Channel ADC inputs\newline Watchdog&
                10&
                540&
                Microchip Technology&\multirow{2}{3cm}{Microcontroller for controlling the PnP Devices}\\
                \cline{1-5}
                ATMega328P Microcontroller&
                8 Bit\newline 32 KB Flash\newline Advanced RISC Architecture\newline 20 MHz CPU Clock\newline 3 Timers\newline 6 PWM&
                10&
                1,240&
                Microchip Technology&\\\hline
                Relay&
                Current: 16A\newline Coil Voltage: 5V&
                20&
                1,440&
                Amazon*&
                Switching the primitive devices\\ \hline
                DHT11 Humidity and Temperature Sensor&
                Voltage: DC 5V\newline
                Digital Output\newline
                Humidity Range: 20-90RH\newline
                Temperature range:	0-60
                &
                10&
                860&
                Evelta*&
                \multirow{3}{3cm}{Sensors for developing Sensor Aggregator modules}\\
                \cline{1-5}
                HC-SR501 PIR Sensor&&2&858&Evelta*&\\
                \cline{1-5}
                GY-30 Light Intensity Sensor&&10&1,150&Evelta*&\\
                \hline
                Jumper Wires&&5&1,160&Amazon*&\multirow{3}{*}{Prototyping}\\
                \cline{1-5}
                Breadboard&
                &
                10&
                1290&
                Amazon*&
                \\
                \cline{1-5}
                Printed Circuit Boards&
                &
                20&
                400&
                J. B. Electronics&\\
                \hline
                Philips Base B22 9-Watt LED Bulb&
                &
                1&
                669&
                Philips&
                \multirow{2}{3cm}{Consumables for developing control methodology for PnP Modules}\\
                \cline{1-5}
                Crompton Hill Briz 1200mm Ceiling Fan&
                Sweep: 1200mm\newline Bearing: Double ball bearing\newline
                Blades: 3\newline Speed: 370 RPM\newline&
                3&
                4,347&
                Crompton Greaves&\\
                \hline
                Circuit Components&&&2,500&J. B. Electronics&Capacitors, MOSFETS etc. required to develop driver circuitry for PnP and Sensor Aggregator modules\\
                \hline
                Miscellaneous Expenses&&&5,000&&Contingency reserve expenses\\
                \hline
                \multicolumn{3}{|l|}{\textbf{Total}}&\textbf{30,034}&&\\
                \hline
            \end{longtable}
        \end{center}
        
    \end{document}
    