 \chapter{Design Methodology}
        \section{Methodology}
        \textbf{A. Development of driver circuit} \\
        
        To introduce software based control, microcontroller based actuators modules are to be attached in the power supply lines of the devices (fan and lighting system control).\\ 
        
        \textbf{A.1. Study of control methodology} The development of actuator modules designed for each device to be automated will be using plug and play methods so as to convert existing devices into smart ones. The actuators will operate and automate the appliances based on the commands received from the microprocessor.\\
        
        \textbf{A.2. Actuator circuit design} 
        \begin{itemize}
        \item \textbf{Incandescent Lamp} The use of pulse width modulated signals that drives a bidirectional control switch i.e., Triac which is connected to the Lamp and thereby with the AC mains. As the width of the PWM is altered the voltage across the lamp will be varied and hence the brightness of the lamp will be controlled.
         \item \textbf{Fans} Speed regulation of fans will be done using the Triac to reduce the energy losses that were occurring by the use of conventional voltage controller. A snubber circuit is connected in parallel with the triac in order to protect it against reverse breakdown.\\
         
         
         Now the designed circuit of actuator modules is designed in such a way that there would be no need to interfere with the existing circuitry of the appliances. The existing switches will be simply upgraded by the smart switch boards providing automated control of each appliance over the internet.
        \end{itemize}
        \textbf{B.  Design of the software for the Smart Switch Board} \\
        
        The smart switch board needs a central hub for their communication and management. It will receive the inputs from the zero crossing detection circuit and give an optimized output pulse to the switch after processing it using the aforementioned algorithms.\\
        
        The microcontroller ESP8266 is programmed on NodeMCU platform. The coding is done in C++ language for the microcontroller in order to generate the firing pulses accordingly to operate the driver circuits.\\
        
        \textbf{C. Development of a user interface to enable user interaction with the system} \\
        
        Applications will be developed for the most common mobile platforms, iOS and Android. A cloud-based web app will also be deployed for users to operate from laptops and desktops. The user will be able to use these apps to connect directly to the server hosted on the Home Automation Unit in their homes without any middleware services ensuring their security and privacy.\\
        
        The programming of the Android Application is done in Java XML. The MQTT broker (CloudMQTT.com) is being used for machine-to-machine internet of things connectivity. It works on a publish/subscribe methodology, and is a lightweight messaging protocol developed on top of the HTTP.\\
        
        \begin{figure}[H]
        	\centering
        	\includegraphics[width=0.5\textwidth]{photos/theory/app_screen1.jpg}
        	\caption{Room Selection Screen}
        \end{figure}
	    \begin{figure}[H]
	        \centering
	        \includegraphics[width=0.5\textwidth]{photos/theory/app_screen2.jpg}
	        \caption{Device Control Screen}
	    \end{figure}
        
        
        \section{Flow Chart}
        
        \begin{figure}[H]
        	\centering
        	\includegraphics[width=\textwidth]{photos/gantt/FlowChart.png}
        	\caption{Logic flow for operation of the Smart Switchboard}
        \end{figure}
        
        
        
        
        \section{Mathematical analysis and calculations}
        \subsection{Power Supply}
       \begin{equation*}
      V_{inp}=230V=V\textsubscript{p} (primary\,voltage) 
      \end{equation*} 
      \begin{equation*}
       V_{s}(secondary\,voltage)=13.74V
             \end{equation*} 
             \begin{equation}
             V_{cp}(peak\,capacitor\,voltage)=13.74*\sqrt{2}-2*0.7 = 18.03V
            \end{equation}
            
            
            From the data sheet(attached) of Voltage Regulator(7805), the minimum voltage at the Input Terminal must be above 7V.\\
           	As we know the discharging equation of capacitor is :
             \begin{equation}
             V=V_{cp}*e^\frac{-t}{T}
             \end{equation}
            \begin{equation}
            7=18.03*e^\frac{-5*10^{-3}}{T}
             \end{equation}
             By Solving,
             \begin{equation*}
             T=5.28ms
             \end{equation*}
              \begin{equation*}
             R*C=5.28ms
             \end{equation*}
             From Diode Bridge Rectifier,
              \begin{equation*}
             V(drop)=1.4V=I*R
             \end{equation*}
              \begin{equation*}
            I=500mA_{(load\,current)}
             \end{equation*}
       	By solving, 
       	\begin{equation}
             R=2.8{\si{\ohm}}
             \end{equation}
             Putting in eqn 3.7,
             \begin{equation}
             C=1880{\si\micro}F
             \end{equation}
             So, we have used slightly larger value of capacitor i.e. 2200{\si\micro}F.
             \subsection{Zero Crossing Circuit}
             From data sheet(attached) of A4N25 Transistor based Octocoupler:
             For internal Diode to turn ON -
             \begin{equation}
             I(max.\,permissible\,limit)=60mA
             \end{equation}
             \begin{equation}
            Rectified\,Voltage,V=5.21V
             \end{equation}
             \begin{equation}
            R_{connected}=165{\si\ohm}
             \end{equation}
             \begin{equation}
             I_{actual}=\frac{V}{R_{connected}}=31.57mA
             \end{equation}
        
        \subsection{Triac Circuit}
             From the Triac Circuit,
             \begin{equation}
            V_{0\,rms}=\sqrt{\frac{1}{2{\pi}}(\int^{\pi}_{\alpha}{{(V_msin\theta)}^2}d\theta+\int^{2\pi}_{\alpha+\pi}{{(V_msin\theta)}^2}d\theta}
             \end{equation}
            On solving, the output rms voltage of Triac is:
            \begin{equation}
            V_{0\,rms}=\frac{V_m}{\sqrt{2}}[\sqrt{\frac{1}{\pi}((\pi-\alpha)+\frac{sin2\alpha}{2})}]
             \end{equation}
             Now, for different firing angle, the output voltage across load would be different as:
             for $\alpha$=\ang{0}
             \begin{equation*}
             V_{o\,rms}=\frac{V_m}{\sqrt{2}}
             \end{equation*}
             for $\alpha$=\ang{30}
             \begin{equation*}
            V_{o\,rms}=\frac{V_m}{\sqrt{2}}*0.98
             \end{equation*}\
             \& so on...
             
             \subsection{Optocoupler}
             From datasheet of MOC3041 Optocoupler:
             \begin{equation}
             I(max.\,permissible\,limit)=60mA
             \end{equation}
             Now, \\
             \begin{equation}
             Voltage\,applied,V=3.3V
             \end{equation}
             Hence,
             \begin{equation}
             R_{reqd.}=\frac{V}{I}=\frac{3.3V}{60mA}=55{\si\ohm}
             \end{equation}
             
             \subsection{AC Fan}
             From datasheet(attached) of ac axial fan:
             \begin{equation*}
            L=3H,\,R=1.67K{\si\ohm}
             \end{equation*}
             So,
             \begin{equation}
             X_L=w*L=3*2\pi*50=942.477
             \end{equation}
             \begin{equation*}
             tan{\theta}=\frac{X_L}{R}
             \end{equation*}
             \begin{equation*}
             \theta=tan^{-1}(\frac{X_L}{R})
             \end{equation*}
             \begin{equation*}
             \theta=tan^{-1}(\frac{94204777}{1.67*1000})
             \end{equation*}
             \begin{equation}
             \theta=29.43^\circ
             \end{equation}
             Now, for the circuit to start operating-\\
             \begin{equation*}
             Firing\,angle(\alpha)>=\theta 
             \end{equation*}
             Hence,
             \begin{equation}   
             \alpha>\ang{29.43}
             \end{equation}
             
             \subsection{Power Rating of the Smart Switchboard}
             
             Keeping in mind the trace widths used for the high power component of our PCB, the following maximum ratings for the Smart Switchboard have been calculated.\\             
             Trace Width for High Power side, 
             \begin{equation*}
             t = 1.5mm
             \end{equation*}
             By referring to a professional PCB manufacturer, BITTELE (www.7pcb.com), the following maximum current limit was determined using their Trace Width Calculator.
             \begin{equation*}
             I = 5.3A
             \end{equation*}
             Our circuit operates at a rated RMS voltage of 230V. Hence,
             \begin{equation*}
    	         P_{max} = 5.3A * 230 V
             \end{equation*}
             \begin{equation}
	             P_{max} = 1219 W
             \end{equation}
             
               \newpage
        \section{Circuit Diagram}
        
        	Following are the schematics prepared on the Autodesk Eagle schematic and PCB design software. The simulation for same was performed using OrCAD (PSpice).
        
        	\begin{figure}[H]
        		\centering
        		\includegraphics[width=0.9\textwidth]{photos/schematics/5V_DC.png}
        		\caption{5V DC Power Supply}
        	\end{figure}
	        \begin{figure}[H]
	        	\centering
	        	\includegraphics[width=0.9\textwidth]{photos/schematics/zero_cross.png}
	        	\caption{Zero Crossing Detection Circuit}
	        \end{figure}
	        \begin{figure}[H]
	        	\centering
	        	\includegraphics[width=0.7\textwidth]{photos/schematics/esp8266.png}
	        	\caption{The esp8266 Microcontroller}
	        \end{figure}
		    \begin{figure}[H]
		    	\centering
		    	\includegraphics[width=0.9\textwidth]{photos/schematics/switch.png}
		    	\caption{Power Circuit for On/Off Switch}
		    \end{figure}
		    \begin{figure}[H]
		    	\centering
			    \includegraphics[width=0.9\textwidth]{photos/schematics/bulb.png}
			    \caption{Power Circuit for Incadescent Bulb}
		    \end{figure}
			\begin{figure}[H]
				\centering
				\includegraphics[width=0.9\textwidth]{photos/schematics/fan.png}
				\caption{Power Circuit for Fan}
			\end{figure}
		\newpage
		
        \section{Hardware Design}
        
        	The prepared schematics after verification of results in the sofftware simulations were laid out on a PCB using Eagle CAD. This enabled us to minimise an otherwise complex and cumbersome circuit into a simplified design which could be installed inside a small enclosure resembling a switchboard in everyday homes.
        
	        \begin{figure}[h!]
	        	\centering
	        	\includegraphics[width=\textwidth]{photos/hardware/PCBDesign.png}
	        	\caption{PCB Layout for the design}
	        \end{figure}
		 
        \section{Hardware System}
        
        	The prepared PCB was designed and ordered using a PCB prototyping facility. The components were acquired from a local retailer and soldered to create a working prototype of our concept.\\
        	
        	The prototype went through repeated tests, and after numerous occasions of failure and fine-tuning, a working prototype was prepared for demonstration. \\
        
	        \begin{figure}[H]
	        	\centering
	        	\includegraphics[width=0.8\textwidth]{photos/hardware/sb_on.jpg}
	        	\caption{Demonstration of the complete project}
	        \end{figure}

			\begin{figure}[H]
				\centering
				\includegraphics[width=0.8\textwidth]{photos/hardware/sb_off.jpg}
				\caption{Demonstration of the complete project}
			\end{figure}  
			\begin{figure}[H]
				\centering
				\includegraphics[width=0.8\textwidth]{photos/hardware/sb_in.jpg}
				\caption{Demonstration of the complete project}
			\end{figure}  