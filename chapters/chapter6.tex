\chapter{Project Metrics}
        \section{Challenges faced and Troubleshooting}
        
        Various challenges were faced during the simulations and hardware design of the Smart Switch Board-
        \begin{enumerate}
			\item \textbf{Working with 230V power supply}
			
			One of our major challenges was to tackle the 230V supply as majority of the components got damaged due to excess currents in the lines, inadequate ratings of the devices used in the power circuit, shorting of lines during measurements of voltages and currents or improper soldering.
			
			\item \textbf{Familiarization with various circuit simulation platforms}
			
			For the simulation of control circuits, we were required to have a good command over various simulation platforms. The circuit simulation softwares used were Orcad Cadence, Matlab R2015a and EasyEda.
			
			\item \textbf{Grasp over programming tools and android applications}
			
			For software programming of the controllers, single board computers and android application development Java, Android Studio and C++ platforms were learned.
			
			\item \textbf{Proper grounding}
			
			We damaged many circuits in the process due to earthing  problems. In one case, the analog ground of the circuit was shorted with the digital ground of the DSO. As both of them were at different potentials an overflow of current was observed which lead to damaging of the analog circuit.
			
			\textbf{Solution:} Proper grounding of the circuits is required ensuring the excess charges in the circuit during faults flow through ground to prevent damage.
			
			\item \textbf{Problems of Voltage Fluctuations}
			
			The voltage fluctuations in the transformer were observed upon applying load to the circuit.
			
			This lead to very low values of currents at the load ends and the devices could not be operated.
			
			These fluctuations occur upon exceeding a permitted range of per unit impedance (upto 5%).
			
			\textbf{Solution:} Higher current ratings of transformers were taken(230V, 1A) in order to obtain steady output at the load end.
			
			\item \textbf{Ratings of the components used}
			
			The calculations were done before hand of the circuit parameters for analysing the circuit. The component ratings of some devices that came out from the calculations were not available in the market.
			
			\textbf{Solution:} Either using series/parallel combination of devices or recalculation of the values of component ratings were done to achieve the desired results.
			
			\item \textbf{Difficulty in analysis:}
			
			The Power Electronics circuit was difficult to analyse. The values of  filter circuits, current limiting resistors and waveform analysis were quite cumbersome. Snubber circuit designed for inductive loads was difficult to analyse as it involved second order differential equations.
			
			\item \textbf{Fine Soldering Requirements:}
			
			The circuit consisted of too many connections and also in order to make it appear as a single switch board, the connections were done very close to one another.
			
			\textbf{Solution:} This required very fine soldering of the circuit and also it took long hours to plan the placement and solder the circuit elements to minimize number of connections and obtain a neat and presentable circuit.
			
			\item \textbf{Smoothing Capacitor values:}
			
			Maximum permissible values of Smoothing capacitors required to provide ripple free supply had to be calculated, exceeding which the capacitors would generate sparks in the circuit and also provide excessive boost to the circuit voltage.
			
			\item \textbf{Operation of full wave rectifier under no load:}
			
			The full wave rectifier shows a distorted output when connected without load. This happens because on no load condition the current drawn by the circuit is zero and hence the other pair of diodes do not actually go into conduction mode thus generating either a half wave rectified output or the CRO was showing abnormal waveform at the output.  
			
			\textbf{Solution:} A resistive load of 33k was applied to the  output terminals of the full wave rectifier circuit for measurement of the voltage waveform across it.  
			
			\item \textbf{Matching of auto-reload of microcontroller with zero crossing signals:}
			
			The zero crossing signals to the microcontroller were provided at a time interval of 10ms which could not be sometimes not interpreted by the microcontroller. This was due to defective microcontroller used in the testing phase.
			
			\textbf{Solution:} It was replaced by a new controller and the problem was rectified.
			
			
			\end{enumerate}
        \section{Relevant Subjects}
                 	\begin{table}[h!]
	        	\begin{tabular}{| l | p{4cm} | p{7.5cm} |}
	        		\hline
	        		\textbf{Subject Code} & \textbf{Subject Code} & \textbf{How was the subject used?}\\
	        		\hline
	        		UTA007 & Computer Programming-I & Used in Microcontroller Programming for operating the driver circuits. \\
	        		\hline
	        		UTA009 & Computer Programming-II & Used in Programming of Single Board Computer.\\
	        		\hline
	        		UTA011 & Engineering Design-III & Operation of Fan circuitry.\\
	        		\hline
	        		UEE301 & Direct Current Machines and Transformers & Transformer used for stepping down the supply for input to the controller. \\
	        		\hline
	        		UEE505  & Analog and Digital Systems & Analysis \& calculations of formed circuit for device control.\\
	        		\hline
	        		UEE401 & Alternating Current Machines & Voltage control of single phase  capacitor run induction motor for fan circuit.\\
	        		\hline
	        		UEE504  & Power Electronics & For understanding the power electronic (Triac, Optoisolators, etc.) device characteristics and operation used in control circuitry.\\
	        		\hline
	        		      		
	        		UEI609 & Fundamentals of Microprocessors and Microcontrollers & For understanding basics of microcontrollers. \\
	        		\hline
	        		UEE603  & Switch Gera \& Protection & Use of fuses for overcurrent protection of devices \&  optocouplers for Isolation of circuit.\\ 
	        		\hline
	        		UEE801  & Electric Drives & Input voltage control of Ac motors for speed control of fan. \\
	        		\hline
	        		
	        		\end{tabular}
	        		\end{table}	
        	
        \section{Interdisciplinary Aspect}
        	This project consists of extensive multidisciplinary efforts.\\
        	The user application developed for providing a user interface is a component of Software Engineering.\\
        	The enclosure for the smart switchboard was designed with inputs from our friends in Mechanical Engineering.\\
        	
        \section{Components Used}
	        \subsection{Software Used}
	        \begin{itemize}
	        	\item MATLAB
	        	\item OrCad (PCB Design software)
	        	\item Autodesk Eagle (PCB Design software)
	        	\item Pspice (Electrical simulation software)
	        	\item MQTT Dashboard (Android Application)
	        	\item CloudMQTT.com (A cloud based secure MQTT Broker)
	        \end{itemize}
	        
	        %Hardware to be used: (Details specifications and purpose of each equipment)
	        \subsection{Hardware Used}
	        \begin{table}[h!]
	        \begin{tabular}{| p{2cm} | p{7cm} | p{3cm} |}
	       	\hline
	        \textbf{Sr. No.} &\textbf{Component Used} &\textbf{Quantity} \\ \hline
	        1&OptoCoupler(MOC 3021)&2 \\ \hline
	        2&OptoCoupler(MOC3041)&1\\ \hline
	        3&Triac(BT136)&3 \\ \hline
	        4&Volatage Regulator(LM 7805)&1\\ \hline
	        5&Diode(1N 4007A)&8\\ \hline
	        6&OptoCoupler(A4N25)&1 \\ \hline
	        7&Capacitor(2200{\si\micro}FF)&1 \\ \hline
	        8&Capacitor(470{\si\micro}FF)&1 \\ \hline
	        9&Resistor(0.54w,54{\si{\ohm}})&3 \\ \hline
	        10&Resistor(0.5w,165{\si{\ohm}})&1 \\ \hline
	        11&Resistor(1w,390{\si{\ohm}})&3 \\ \hline
	        12&Resistor(1w,470{\si{\ohm}})&3 \\ \hline
	        13&Resistor(39{\si{\ohm}})&3 \\ \hline
	        14&Capacitor(0.01{\si\micro}FF)&2\\ \hline
	        15&2 Pin PCB Mount Connector&5\\ \hline
	        16&PCB&1 \\ \hline
	        17&Micro Controller(ESP8266)& 1 \\ \hline  
	        \end{tabular}
        	\centering
	        \caption{List of coomponents used in the finished product}
	        \end{table}
        
        \section{Team Assessment Matrix}
	        \begin{table}[h!]
	        	\begin{tabular}{| l | c | c | c | c |}
	        		\hline
	        		& \multicolumn{4}{c|}{\textbf{Evaluation of}} \\\hline
	        		\textbf{Evaluation by} & Satyam Kumar & Shubham Gupta & Stuti Sidhu & Swanav Swaroop \\
	        		\hline
	        		Satyam Kumar   & 5.0 & 4.5 & 4.5 & 4.5\\
	        		\hline
	        		Shubham Gupta  & 4.5 & 5.0 & 4.5 & 4.5\\
	        		\hline
	        		Stuti Sidhu    & 4.5 & 4.5 & 5.0 & 4.5 \\
	        		\hline
	        		Swanav Swaroop & 4.5 & 4.5 & 4.5 & 5.0 \\ [1ex] 
	        		\hline
	        	\end{tabular}
		        \caption{Team Assessment Matrix}
	        \end{table}
        \section{Work Schedule}
        	\begin{table}[H]
        		\begin{figure}[H]
        			\centering
        			\includegraphics[width=\textwidth]{photos/gantt/SatyamGC.png}
        		\end{figure}
	        	\caption{Work Schedule for Satyam}
        	\end{table}
			\begin{table}[H]
				\begin{figure}[H]
					\centering
					\includegraphics[width=\textwidth]{photos/gantt/ShubhamGC.png}
				\end{figure}
				\caption{Work Schedule for Shubham}
			\end{table}
			\begin{table}[H]
				\begin{figure}[H]
					\centering
					\includegraphics[width=\textwidth]{photos/gantt/StutiGC.png}
				\end{figure}
			\caption{Work Schedule for Stuti}
			\end{table}
			\begin{table}[H]
				\begin{figure}[H]
					\centering
					\includegraphics[width=\textwidth]{photos/gantt/SwanavGC.png}
				\end{figure}
				\caption{Work Schedule for Swanav}
			\end{table}
\newpage
        \section{Student Outcome Mapping}
        	\begin{table}[h!]
        		\begin{tabular}{|m{0.75cm}|m{2.75in}|m{2.75in}|}\hline
        		A1.&Applied mathematics (partial differentiation, vector calculus, linear algebra, complex variables, Laplace transform, probability, statistics, discrete mathematics etc.) to obtain analytical, numerical and statistical solutions.& Used in calculating the values of various devices used in the circuit w.r.t circuit ratings.\\\hline
        		A2 & Demonstrate and apply knowledge of fundamentals, scientific and engineering principles towards solving engineering problems.& The project requires analysis of the wave form across the output \& their variation upon changing the firing angle.The project demonstrates the use of microcontrollers, Ac motors, triacs.\\\hline
        		B2 & Utilize suitable hardware equipment for data collection.& The project used digital signal oscilloscope for data acquisation. \\\hline
          		D1 & Share responsibility and information schedule with others in team.& Each \& every member in the team was assigned a particular section of the project \& they discussed the project status with every member.\\\hline
        		E1 & Classify information to identify engineering problems.& Various practical problems were encountered during the hardware assembly.\\\hline
        		G1 & Prepare and present variety of documents such as project or laboratory reports and inspection reports with discipline specific standards.& The team members presented their idea to the mentor \& the report has been prepared following proper IEEE standards used in the design. \\\hline
        		I1 & Able to use resources to adopt new technologies not included in curriculum.& We have used OrCAD Cadence for design \& simulation of circuits. Eagle CAD was used for design of the PCB.\\\hline
        		J2 & Recognize the impact of engineering decisions reduces on energy resources and environment.& The devices used consume negligible power \& the control strategy is efficient in reducing power consumption.\\\hline
        		K3 & Able to analyze engineering problems using software tools.& The results \& calculations have been compared with the software used. \\\hline
        		\end{tabular}
        	\end{table}
    