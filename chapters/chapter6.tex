\chapter{Project Metrics}
        \section{Challenges faced and Troubleshooting}
        
        Various challenges were faced during the simulations and hardware design of the Smart Switch Board-
        \begin{enumerate}
			\item Our major challenge was to tackle the 230V supply as majority of the components were damaged due to the shorting of lines.
			
			\item For the simulation of control circuits, we were required to get comfortable with various simulation platforms (Orcad Cadence, Matlab R2015a, EasyEda).
			
			\item For software programming of the controllers and single board computers and android application development Java, Android Studio and C++ platforms were learned.
			
			\item Proper grounding of the circuits had to be done with care. Various switch boards were compensated and circuit brakers were tripped in the testing phase.
			
			\item Voltage fluctuations were observed in the transformer upon application of load to the circuit.
			
			\item  Components of calculated ratings were not available in the market. Various individual circuits had to be made for achieving the desired values.
			
			\item  Calculations and circuit analysis was difficult of the control circuitry.
			
			\item The circuit was very concise very fine soldering was done which took long hours to conduct.
			
			\item Maximum permissible values of Smoothing capacitors required to provide ripple free supply had to be calculated, exceeding which the capacitors would generate sparks in the circuit and also provide excessive boost to the circuit voltage.
			
			\item The full wave rectifier shows a distorted output when connected without load.
			
			\item The matching of the auto reload of time of microcontroller and zero crossing circuit signals had to be done.
			
			\item Capacitors of higher ratings had to be discharged first before implementing in the circuit, which generated sparks when their ends were shorted.
			
			\item Isolated DSO needed to be used for observing the waveform generated by the output load, because of the phase shift in both the device voltages. Connecting the DSO tripped the circuit on multiple instances, often inducing sparks and damaging the components.
			\end{enumerate}
        \section{Relevant Subjects}
                 	\begin{table}[h!]
	        	\begin{tabular}{| l | p{4cm} | p{7.5cm} |}
	        		\hline
	        		\textbf{Subject Code} & \textbf{Subject Code} & \textbf{How was the subject used?}\\
	        		\hline
	        		UTA007 & Computer Programming-I & Used in Microcontroller Programming for operating the driver circuits. \\
	        		\hline
	        		UTA009 & Computer Programming-II & Used in Programming of Single Board Computer.\\
	        		\hline
	        		UTA011 & Engineering Design-III & Operation of Fan circuitry.\\
	        		\hline
	        		UEE301 & Direct Current Machines and Transformers & Transformer used for stepping down the supply for input to the controller. \\
	        		\hline
	        		UEE505  & Analog and Digital Systems & Analysis \& calculations of formed circuit for device control.\\
	        		\hline
	        		UEE401 & Alternating Current Machines & Voltage control of single phase  capacitor run induction motor for fan circuit.\\
	        		\hline
	        		UEE504  & Power Electronics & For understanding the power electronic (Triac, Optoisolators, etc.) device characteristics and operation used in control circuitry.\\
	        		\hline
	        		      		
	        		UEI609 & Fundamentals of Microprocessors and Microcontrollers & For understanding basics of microcontrollers. \\
	        		\hline
	        		UEE603  & Switch Gera \& Protection & Use of fuses for overcurrent protection of devices \&  optocouplers for Isolation of circuit.\\ 
	        		\hline
	        		UEE801  & Electric Drives & Input voltage control of Ac motors for speed control of fan. \\
	        		\hline
	        		
	        		\end{tabular}
	        		\end{table}	
        	
        \section{Interdisciplinary Aspect}
        	This project will consist of extensive multidisciplinary efforts going further.
        	A Load Optimization algorithm will be generated fo future autonomous functionality using a Deep Reinforcement Learning model which is primarily a topic of interest in Computer Science.
        	The wireless communication among the sensors, devices and the HAU are subjects of Electronics and Communication Engineering.\\
        \section{Components Used}
	        \subsection{Software Used}
	        \begin{itemize}
	        	\item MATLAB
	        	\item OrCad  (PCB Design software)
	        	\item Pspice (Electrical simulation software)
	        	\item MQTT Dashboard (Android Application)
	        	\item CloudMQTT.com (A cloud based secure MQTT Broker)
	        \end{itemize}
	        
	        %Hardware to be used: (Details specifications and purpose of each equipment)
	        \subsection{Hardware Used}
	        \begin{table}[h!]
	        \begin{tabular}{| p{2cm} | p{7cm} | p{3cm} |}
	       	\hline
	        \textbf{Sr. No.} &\textbf{Component Used} &\textbf{Quantity} \\ \hline
	        1&OptoCoupler(MOC 3021)&2 \\ \hline
	        2&OptoCoupler(MOC3041)&1\\ \hline
	        3&Triac(BT136)&3 \\ \hline
	        4&Volatage Regulator(LM 7805)&1\\ \hline
	        5&Diode(1N 4007A)&8\\ \hline
	        6&OptoCoupler(A4N25)&1 \\ \hline
	        7&Capacitor(2200{\si\micro}FF)&1 \\ \hline
	        8&Capacitor(470{\si\micro}FF)&1 \\ \hline
	        9&Resistor(0.54w,54{\si{\ohm}})&3 \\ \hline
	        10&Resistor(0.5w,165{\si{\ohm}})&1 \\ \hline
	        11&Resistor(1w,390{\si{\ohm}})&3 \\ \hline
	        12&Resistor(1w,470{\si{\ohm}})&3 \\ \hline
	        13&Resistor(39{\si{\ohm}})&3 \\ \hline
	        14&Capacitor(0.01{\si\micro}FF)&2\\ \hline
	        15&2 Pin PCB Mount Connector&5\\ \hline
	        16&PCB&1 \\ \hline
	        17&Micro Controller(ESP8266)& 1 \\ \hline  
	        \end{tabular}
        	\centering
	        \caption{List of coomponents used in the finished product}
	        \end{table}
        
        \section{Team Assessment Matrix}
	        \begin{table}[h!]
	        	\begin{tabular}{| l | c | c | c | c |}
	        		\hline
	        		& \multicolumn{4}{c|}{\textbf{Evaluation of}} \\\hline
	        		\textbf{Evaluation by} & Satyam Kumar & Shubham Gupta & Stuti Sidhu & Swanav Swaroop \\
	        		\hline
	        		Satyam Kumar   & 5.0 & 4.5 & 4.5 & 4.5\\
	        		\hline
	        		Shubham Gupta  & 4.5 & 5.0 & 4.5 & 4.5\\
	        		\hline
	        		Stuti Sidhu    & 4.5 & 4.5 & 5.0 & 4.5 \\
	        		\hline
	        		Swanav Swaroop & 4.5 & 4.5 & 4.5 & 5.0 \\ [1ex] 
	        		\hline
	        	\end{tabular}
		        \caption{Team Assessment Matrix}
	        \end{table}
        \section{Work Schedule}
        \begin{table}
        	 {\centering
        		\begin{ganttchart}[expand chart=\textwidth]{1}{12}
        			\gantttitle{2018}{12} \\
        			\gantttitlelist[]{1,...,12}{1} \\
        			\ganttbar{Literature Survey}{4}{4} \\
        			\ganttgroup{WP 1}{4}{6} \\
        			\ganttbar{WP1.1}{4}{4} \\
        			\ganttlinkedbar{WP1.2}{5}{6} \ganttnewline
        			\ganttmilestone{Completion}{6} \ganttnewline
        			\ganttgroup{WP 2}{4}{6} \\
        			\ganttbar{WP2.1}{4}{5} \\
        			\ganttlinkedbar{WP2.2}{5}{6} \ganttnewline
        			\ganttmilestone{Completion}{6} \ganttnewline
        			\ganttgroup{WP 3}{4}{7} \\
        			\ganttbar{WP3.1}{4}{7} \ganttnewline
        			\ganttmilestone{Completion}{7} \ganttnewline
        			\ganttgroup{WP 4}{8}{11} \\
        			\ganttbar{WP4.1}{8}{11} \\
        			\ganttbar{WP4.2}{10}{11} \ganttnewline
        			\ganttmilestone{Completion}{11} \ganttnewline
        			\ganttbar{Integration}{12}{12}
        			\ganttlink{elem2}{elem3}
        			\ganttlink{elem6}{elem7}
        			\ganttlink{elem9}{elem10}
        			\ganttlink{elem12}{elem14}
        			\ganttlink{elem13}{elem14}
        			
        			\ganttlink{elem3}{elem15}
        			\ganttlink{elem7}{elem15}
        			\ganttlink{elem10}{elem15}
        			\ganttlink{elem14}{elem15}
        		\end{ganttchart}
        	}
        \caption{Work Schedule for the complete project}
        \end{table}

        \section{Student Outcome Mapping}
        	\begin{table}[h!]
        		\begin{tabular}{|m{0.75cm}|m{2.75in}|m{2.75in}|}\hline
        		A1.&Applied mathematics (partial differentiation, vector calculus, linear algebra, complex variables, Laplace transform, probability, statistics, discrete mathematics etc.) to obtain analytical, numerical and statistical solutions.& Used in calculating the values of various devices used in the circuit w.r.t circuit ratings.\\\hline
        		A2 & Demonstrate and apply knowledge of fundamentals, scientific and engineering principles towards solving engineering problems.& The project requires analysis of the wave form across the output \& their variation upon changing the firing angle.The project demonstrates the use of microcontrollers, Ac motors, triacs.\\\hline
        		B2 & Utilize suitable hardware equipment for data collection.& The project used digital signal oscilloscope for data acquisation. \\\hline
          		D1 & Share responsibility and information schedule with others in team.& Each \& every member in the team was assigned a particular section of the project \& they discussed the project status with every member.\\\hline
        		E1 & Classify information to identify engineering problems.& Various practical problems were encountered during the hardware assembly.\\\hline
        		G1 & Prepare and present variety of documents such as project or laboratory reports and inspection reports with discipline specific standards.& The team members presented their idea to the mentor \& the report has been prepared following proper IEEE standards used in the design. \\\hline
        		I1 & Able to use resources to adopt new technologies not included in curriculum.& We have used ORCAD CADENCE for design \& simulation of circuits.\\\hline
        		J2 & Recognize the impact of engineering decisions reduces on energy resources and environment.& The devices used consume negligible power \& the control strategy is efficient in reducing power consumption.\\\hline
        		K3 & Able to analyze engineering problems using software tools.& The results\& calculations have been compared with the software used. \\\hline
        		\end{tabular}
        	\end{table}
    