\chapter{Theory, Standards and Constraints}
        \section{Theory}
        
        \subsection{Phase Angle Control}
        	
        	Phase angle control is a simple AC-AC conversion technique used to convert an input AC RMS Voltage to a reduced RMS value. It uses a low frequency switch to chop the AC sine wave. The reduced output voltage is controlled using a quantity known as the Firing Angle. The firing angle controls the output of the sine wave using delayed firing of the switching elements (triac, SCR etc.) ensuring the magnitude of AC rms signal is reduced. The output voltage is calculated by the determining the area under the curve, the subsequent increase in the firing angle leads to decrease in value of rms and hence the brightness of light or speed of fan can be controlled.\\
        	
        	
        	\begin{wrapfigure}{l}{0.45\textwidth}
        		\includegraphics[width=0.45\textwidth]{photos/theory/rms_output.png}
        		\caption{Variation in RMS Voltage Output with the Firing Angle}
        	\end{wrapfigure}
        	There are a variety of \textbf{advantages} of this approach:\\
        	\begin{itemize}
        		\item It uses simple voltage control to operate the devices at varied brightness and speeds.
        		\item This allows the use of a low frequency switch, in our case a Triac. These low frequency switches have much higher ratings and can be applied directly to the power circuit.
        		\item This allows for a very fine control over the voltage values and thus numerous steps for variation in device parameters can be provided.
        	\end{itemize}
        	
        	\textbf{Disadvantages} of the approach can be:
        	\begin{itemize}
        		\item Harmonic inclusion in the AC waveform leading to shortening of life of devices.
        		\item Reliability of the controlling switches is reduced as the switching does not occur at zero voltages and currents.
       		\end{itemize}
       	
        	The following graphs show the variation in Output Voltage waveform with subsequent increase in firing angle. The following data is represented from the above graphs:
        	\begin{enumerate}
        		\item Input Voltage Waveform
        		\item Zero Cross Detection 
        		\item Firing Pulse
        		\item Output Voltage Waveform
        	\end{enumerate}
        
        	\begin{figure}[h!]
        		\centering
        		\includegraphics[width=0.8\textwidth]{photos/theory/pac_30.png}
        		\caption{Variation in voltage waveforms for $\alpha$ = \ang{30}}
        	\end{figure}
	        \begin{figure}[h!]
	        	\centering
	        	\includegraphics[width=0.8\textwidth]{photos/theory/pac_90.png}
	        	\caption{Variation in voltage waveforms for $\alpha$ = \ang{90}}
	        \end{figure}
		    \begin{figure}[h!]
		    	\centering
		    	\includegraphics[width=0.8\textwidth]{photos/theory/pac_120.png}
		    	\caption{Variation in voltage waveforms for $\alpha$ = \ang{120}}
		    \end{figure}
	

			        
        \subsection{5V Power Supply}
        A 5V Power supply is designed to provide uninterrupted mains supply to the microcontroller.\\
        The 230V AC Mains is converted to a 13V AC signal using a Step Down Transformer. The transformer also provides isolation to the circuit. This stepped down voltage is passed through a bridge rectifier to convert it into a pulsating DC output. This DC supply cannot be reliably without converting removing ripples from the output and thereafter regulating the output. Hence, this output is connected to a LM7805 5V Voltage Regulator in parallel with 2200 \si\micro f and a 470 \si\micro f capacitors. This stabilises the output and reduces the input ripples to almost zero.
        
        \begin{figure}[h!]
        	\centering        	\includegraphics[width=0.8\textwidth]{photos/ckt-dgm/5VDCPowerSupply.jpg}
        	\caption{5V DC Power Supply}
        \end{figure}
    
    	\subsection{Zero Crossing Detection Circuit}
    	A zero cross detection circuit is a type of voltage comparator which locates point of zero voltage. In this circuit, a transistor based opto-isolator IC (4N25) is used with rectifier voltage pulse of 100 Hz provided at the input pins. The location of zero point is necessary for following the rectified voltage waveform of the circuit in order to fire the triac at required phase angles. The microcontroller (ESP8266) fires interrupts upon receiving signals from the zero crossing circuit marking the starting point of the counter timer. Now, by calculating the time corresponding to the input firing angle the pulse is fired providing gate signal to the triac, thereby turning it ON.
        
    \subsection{ESP8266}
        \begin{wrapfigure}{r}{0.3\textwidth}
        	\includegraphics[width=0.3\textwidth]{photos/theory/esp8266.jpg}
        	\caption{ESP8266 Wifi Microcontroller}
        \end{wrapfigure}
    
        The ESP8266 WiFi Module has a self contained SoC with an integrated TCP/IP protocol stack which provides it access to a Wi-Fi network. The ESP8266 has the potential of either hosting an application or offloading all Wi-Fi networking functions from another application processor. The ESP8266 module is an extremely cost effective board with a huge, and ever growing, community.\\
        This module has a powerful on-board 32-bit processor with storage capability that allows it to be integrated in various applications. Its high degree of on-chip integration allows for minimal external circuitry, including the front-end module.\\
        
\subsection{Triac}
The Triac is a power electronic device that is used in AC switching application. Since it consists of two thyristors connected back-to-back (in other words, anti-parallel) but on the single piece of silicon so it controls the flow of current in both half cycle of alternating current. \\
\begin{wrapfigure}{r}{0.3\textwidth}
	\includegraphics[width=0.3\textwidth]{photos/theory/bt136.jpg}
	\caption{Triac BT136}
\end{wrapfigure}
Similar to that of the thyristor, the TRIAC also have three terminals namely MT1, MT2 and gate terminal. For the first half cycle MT1 act as anode terminal and MT2 terminal act as cathode terminal and for the next half cycle reverse is the case. However, the gate terminal is fixed and it receives the gating pulse for the TRIAC in both half cycle thus making it capable of controlling the flow of current.\\
Just like thyristor, the TRIAC also needs a minimum value of gate current called as latching current. So the gating pulse must be applied till the anode current go beyond latching current after which the pulse can be removed in order to reduce losses. \\
The TRIAC turns off automatically when the current in the circuit goes below holding current and when the gating pulse is again given in the next half cycle, TRIAC start conducting. This process keeps on going. And thus, the RMS value of output voltage is controlled by changing the delay angle or firing angle of the TRIAC.
Here in our project, the TRIAC is used in Controlling the brightness of lamp and speed of the fan.\\

\subsection{Optocouplers}
\begin{wrapfigure}{l}{0.3\textwidth}
	\includegraphics[width=0.3\textwidth]{photos/theory/moc3021.jpg}
	\caption{Optocoupler MOC3021}
\end{wrapfigure}
Optocoupler is an electronic component that is used to isolate circuits on PCB. It consists of two parts: an LED which receives the pulse and emits the Infrared light and the other is a photosensitive TRIAC which senses the radiation from LED and undergoes conduction. Thus, in turn, it produces a gating signal for the TRIAC of the main power circuit. As soon as the main TRIAC receives this signal it starts conducting. Now the firing pulse given to the TRIAC can be controlled by controlling the pulse firing delay given to the diode of the optocoupler.\\
When the pulse is removed, the currents flowing through the LED of optocoupler becomes zero and the photosensitive device i.e. TRIAC inside the optocoupler stops conducting thus making the whole circuit inoperative. \\
There are further other benefits of using an optocoupler which includes the removal of electrical noise. It also isolates the low-voltage side from the high-voltage side in the circuits.\\
Here in our project we have used two different optocoupler namely MOC3021, which is used to generate the firing pulse for TRIAC which in turn help in controlling the AC output waveform and the other one is MOC3041 which contains a Zero-Crossing Circuit inside it thus making it suitable for ON/OFF application as it generates the firing pulse at the instant when AC signal crosses the zero instant thus reducing the chances of a sudden large rise in current. \\
We also have to take care of the maximum amount of current that the LED inside optocoupler will withstand and hence a proper calculation is made to find out the value of limiting resistor as the peak value input pulse at the terminal of the diode is 3.3v which will be given by the microcontroller in our application.  \\

\subsection{L7805 Voltage Regulator}

\begin{wrapfigure}{r}{0.3\textwidth}
	\includegraphics[width=0.3\textwidth]{photos/theory/lm7805.jpg}
	\caption{5V Voltage Regulator LM7805}
\end{wrapfigure}

The Voltage regulator is used to regulate the input voltage to it to a fixed value. Thus it maintains fixed and hence smooth output voltage regardless of the input voltage.\\
The voltage equivalent to the difference of input and output is released as heat. So the greater is the difference, the more will be the heat that is released. Now in order to dissipate the heat, and to prevent the thermal breakdown of the material used inside the voltage regulator, we do use a heat sink. Thus the heat sink helps in ensuring the proper operation of the voltage regulator.\\  
There are further other benefits of the voltage regulator as it shields and protects the electronic circuitry from any potential damage.\\
Here in our project, we have used voltage regulator 7805 as we need a smooth 5v dc output in order to power up the microcontroller. Also, we need to take care of the input voltage appearing at the terminal of voltage regulator so that the difference in voltage must not be very large. Also, the input voltage must be greater than 7v as according to the datasheet of 7805, the minimum voltage needed at the input terminal to regulate it to 5v is 7v.\\

\textbf{7805 Rating}
\begin{itemize}
	\item Input voltage range 7V- 35V
	\item Current rating Ic = 1A
	\item Output voltage range   V{max}=5.2V ,V{min}=4.8V
\end{itemize}


\subsection{Snubber Circuit}
Due to various conditions that generally occurs in electrical and electronics circuits like overcurrent, overvoltage, a large change in the difference of voltage etc., the components in the circuits may get fails. Now we need to protect our circuit from such anomalies. So for the protection against overcurrent, we do use fuse, heat sinks are used to dissipate the heat generated due to an excessive change in voltage.\\
\begin{wrapfigure}{l}{0.25\textwidth}
	\includegraphics[width=0.25\textwidth]{photos/theory/snubber.jpg}
	\caption{A RC Snubber Circuit}
\end{wrapfigure}

Now there appear some cases where there is a sudden rise in current(di/dt) and a sudden rise in voltage(dv/dt). These spikes are very dangerous and can burn the semiconductor devices connected in the circuit. So in order to protect the circuit against these two conditions of sudden rise we do use snubber circuit. \\
Snubber circuit basically has an inductor in series and capacitor in parallel along with a series resistance. The series inductor protects the circuit against the sudden rise in voltage. This is because of the property of the property of inductor which opposes the sudden rise in current. Also, the capacitor helps in protecting the circuit against the sudden rise in voltage. This is because of the property of capacitor which opposes the sudden rise in the voltage. Also, a series resistance is connected in series with the capacitor to limit the discharging current of the capacitor so that the current in the switch due to capacitor do not lead to an overcurrent condition in through the switch when it is operated. \\

\section{Realistic Constraints}
        \begin{itemize}
      \item  Although the project will be designed with a modular approach with scope of future expansibility, it will be demonstrated on a much smaller scale, due to a lack of infrastructure and budget.

		\item Plug and Play modules with fine level of control will be developed only for Lights, Fans due to lack of budget and time. Rest of the devices will be controlled in only an On/Off state.
		
		\item Testing of the modules will be limited to the devices available in the institute.
        \end{itemize}
        
        \section{Technical Standards Used}
        \begin{description}
        	\item[IEEE802.11] Standard for Wi-Fi
        	\item[P2413] Standard for an Architectural Framework for the Internet of Things (IoT)
        	\item[2755-2017]  IEEE guide for terms and concepts in Intelligent Process Automation.
			\item[IEEE 61850-9-3-2016] International Standard for communication networks and systems for power utility automation
			\item[IEEE 802.15.4] Wireless sensor/control networks.
			\item[IEEE 1016] Software design description.
        \end{description}
