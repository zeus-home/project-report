\chapter{Introduction}
        \section{Introduction}
        \section{Literature Survey}
        	\subsection{National Status Review}
        	\begin{itemize}
        		\item Archana N. Shewale, \textbf{"Renewable Energy Based Home Automation System Using ZigBee"}, International Journal of Computer Technology and Electronics Engineering (IJCTEE), 2015\\ \\
        		Archana N. Shewale describes the methodology of renewable energy based home automation in which two things are consider one is energy consumption and another is energy generation. In this, ZigBee is used for monitoring energy consumption of home equipment and power line communication (PLC) is used to monitoring energy generation.
        		\item S. Anusha, \textbf{"Home Automation using ATmega328 Microcontroller and Android application"}, International Research Journal of Engineering and Technology(IRJET), 2015\\ \\
        		S. Anusha describes the design and development of a remote household appliance control system using ATmega328 microcontroller and android mobile through GSM technology.
        		\item J. Chandramohan, \textbf{"Intelligent Smart Home Automation and Security System Using Arduino and Wi-Fi"}, International Journal of Engineering and Computer Science, 2017\\ \\
        		J. Chandramohan provides a low cost-effective and flexible home control and monitoring system with the aid of an integrated micro-web server with internet protocol (IP) connectivity for access and to control of equipment and devices remotely using Android-based smartphone application. generation.
        	\end{itemize}
        	\subsection{International Status Review}
        	\begin{itemize}
        		\item Debraj Basu, \textbf{"Wireless Sensor Network Based DSAda Smart Home: Sensor Selection, Deployment and Monitoring"}, IEEE, 2013\\ \\
        		Debraj Basu details the installation and configuration of unobtrusive sensors in an elderly person?s house - a smart home in the making - in a small city in New Zealand. The overall system is envisaged to use machine learning to analyze the data generated by the sensor nodes.
        		\item Byeongkwan Kang, \textbf{IoT-based monitoring system using tri-level context making model for smart home services"}, IEEE International Conference, 2015\\ \\
        		Kang discusses about acquisition and analysis of sensor data which are going to be used across smart homes. It proposed an architecture for extracting contextual information by analysing the data acquired from various sensors and provide context aware services.
        		\item Jeya Jeya Padmini, \textbf{"Effective Power Utilization and Conservation in Smart Home Using IoT"}, IEEE International Conference, 2015\\ \\
        		Jeya Jeya Padmini discusses about effective power utilization and conservation in smart homes using IoT. It uses cameras for recognizing human activities through image processing techniques.
        		\item Pranay P. Gaikwad, \textbf{"A Survey based on Smart Home System Using Internet of Things"}, IEEE International Conference, 2015\\ \\
        		Pranay P.Gaikwad discusses about challenges and problems arise in smart home systems using IoT and propose possible solutions.
        	\end{itemize}
        	
        \section{Need Analysis}
        
        	The currently available solution and research exhibit these features
        	\begin{itemize}
        		\item The existing market solutions provide on/off switching of the devices. The available solutions provide comparatively lower energy efficiency and also require human intervention to achieve desired conditioning of the environment.
        		\item Some of them also use unsuitable technologies like Power Line Communications, Bluetooth and Ethernet etc. From the present analysis of the existing solutions for automating a room environment, the technologies used suffer from a number of drawbacks. So, these protocols are limited in functionalities when used by an end user in real life.
        	\end{itemize}
        	
        \section{Aim}
        	The project aims to manage and control existing devices through a Home Automation Unit accessible over the internet.
        	The project aims to increase the energy efficiency of a household by converting primitive devices into smart devices and managing them using a self-evolving deep learning model.
        \section{Objectives}
	        \begin{itemize}
	        	\item To detect changes in the environment of rooms using different sensors.
	        	\item To develop plug and play device control techniques to manage home appliances.
	        	\item To optimize energy consumption to increase the efficiency of the household.
	        	\item To design a central processing hub for management of modules and execution of control algorithms.
	        \end{itemize}
        \section{Problem Formulation}
        \section{Deliverables}
        	\begin{itemize}
        		\item Sensor modules to collect room environment parameters.
        		\item Plug and Play actuator modules to control household devices.
        		\item Load Optimization Algorithm based on deep reinforcement learning models
        		\item Home Automation Unit to manage the devices and execute control algorithms and associated mobile applications.		
        	\end{itemize}
        \section{Novelty of work}
        	Higher levels of controllability can be achieved through individual device level control such as for an AC the temperature, modes, swing, fan speed etc. can be controlled. Load optimization will be carried out from the evaluation of the usage patterns of the individual devices thus increasing the energy efficiency. 
    