\chapter{Introduction}
        \section{Introduction}
        
        In the present day scenario, Energy efficiency is the major requirement of any power device (mechanical or electrical). By the introduction of Power Electronics systems electrical systems are aiming towards higher energy efficiency. The losses in these systems are quite low and they enable the user to control the devices to a much larger extent than was possible by using the conventional approach. \\
        
        This project focuses on providing automated control of the existing home appliances by adding a control circuitry to the existing devices. The devices are controlled using wireless communication over the internet by microcontrollers. An app-driven remote control is provided for user interaction. The efficiency of usage can be improved device-by-device. This will enable the user to switch off the devices when idle even remotely and also the power consumptions of devices will be lowered when appliances are not running at their full intensities. Cumulatively, an energy efficient household can be achieved by using such smart switches for controlling appliances used in the house. \\
        
        A smart switchboard is an upgrade to the conventional switches, they use novel Internet of Things (IoT) enabling them to be accessible remotely over the internet. The switches are designed in a modular fashion providing scope for scalability of the project. Based upon the monetary capability of the user and upon introduction of new devices in the household at later stages, the sockets can be easily added or removed at any time in the smart switch. This will enhance the viability and long term relevance of the project.\\ \newpage
        
        \section{Literature Survey}
        	\subsection{National Status Review}
        	\begin{itemize}
        		\item Archana N. Shewale, \textbf{"Renewable Energy Based Home Automation System Using ZigBee"}, International Journal of Computer Technology and Electronics Engineering (IJCTEE), 2015\\ \\
        		Archana N. Shewale describes the methodology of renewable energy based home automation in which two things are consider one is energy consumption and another is energy generation. In this, ZigBee is used for monitoring energy consumption of home equipment and power line communication (PLC) is used to monitoring energy generation.
        		\item S. Anusha, \textbf{"Home Automation using ATmega328 Microcontroller and Android application"}, International Research Journal of Engineering and Technology(IRJET), 2015\\ \\
        		S. Anusha describes the design and development of a remote household appliance control system using ATmega328 microcontroller and android mobile through GSM technology.
        		\item J. Chandramohan, \textbf{"Intelligent Smart Home Automation and Security System Using Arduino and Wi-Fi"}, International Journal of Engineering and Computer Science, 2017\\ \\
        		J. Chandramohan provides a low cost-effective and flexible home control and monitoring system with the aid of an integrated micro-web server with internet protocol (IP) connectivity for access and to control of equipment and devices remotely using Android-based smartphone application. generation.
        	\end{itemize}
        	\subsection{International Status Review}
        	\begin{itemize}
        		\item Debraj Basu, \textbf{"Wireless Sensor Network Based DSAda Smart Home: Sensor Selection, Deployment and Monitoring"}, IEEE, 2013\\ \\
        		Debraj Basu details the installation and configuration of unobtrusive sensors in an elderly person?s house - a smart home in the making - in a small city in New Zealand. The overall system is envisaged to use machine learning to analyze the data generated by the sensor nodes.
        		\item Byeongkwan Kang, \textbf{IoT-based monitoring system using tri-level context making model for smart home services"}, IEEE International Conference, 2015\\ \\
        		Kang discusses about acquisition and analysis of sensor data which are going to be used across smart homes. It proposed an architecture for extracting contextual information by analysing the data acquired from various sensors and provide context aware services.
        		\item Jeya Jeya Padmini, \textbf{"Effective Power Utilization and Conservation in Smart Home Using IoT"}, IEEE International Conference, 2015\\ \\
        		Jeya Jeya Padmini discusses about effective power utilization and conservation in smart homes using IoT. It uses cameras for recognizing human activities through image processing techniques.
        		\item Pranay P. Gaikwad, \textbf{"A Survey based on Smart Home System Using Internet of Things"}, IEEE International Conference, 2015\\ \\
        		Pranay P.Gaikwad discusses about challenges and problems arise in smart home systems using IoT and propose possible solutions.
        	\end{itemize}
        \section{Need Analysis}
        
        	The currently available solution and research exhibit these features
        	\begin{itemize}
        		\item The existing market solutions provide on/off switching of the devices. The available solutions provide comparatively lower energy efficiency and also require human intervention to achieve desired conditioning of the environment.
        		\item Some of them also use unsuitable technologies like Bluetooth, Ethernet etc. From the present analysis of the existing solutions for automating a room environment, the technologies used suffer from a number of drawbacks. So, these protocols are limited in functionalities when used by an end user in real life.
        	\end{itemize}
        	
        \section{Aim}
        	The project aims to manage and control existing devices through a smart switchboard accessible over the internet.
        	        \section{Objectives}
	        \begin{itemize}
	        	\item To develop driver circuits for controlling the devices inside a room
                 \item To develop software programs for the microcontroller to operate the driver circuits.
                  \item To design mobile applications for users to control the devices over the internet.
	        \end{itemize}
        \section{Problem Formulation}
        In the present day scenario, user comfort and convenience is the main target for the product developers. The present conventional devices installed and their manual control does not provide benefit for monitoring and controlling them, when a person is away from home. This leads to large power losses due to unnecessary operation of the devices. This can even cause major damages to life and property if not handled properly. Also, existing commercial solutions in the market provide entire new smart devices which requires users to replace the existing devices incurring huge costs. 
        \section{Deliverables}
        	\begin{itemize}
        		\item A compact and efficient Smart Switch Board is developed.

                 \item A user-friendly mobile application is developed for operating different appliances installed inside the room.
        	\end{itemize}
        \section{Novelty of work}
        Devices can be controlled over the internet through mobile or desktop applications. Due to the use of Power Electronics devices for the voltage control, there are negligible losses incurred. The switches are formed in a modular fashion providing user to add new devices over time. Higher levels of controllability could be achieved through individual device level control.	
    